\documentclass[UTF8]{ctexart}
%\documentclass{article}
\usepackage{graphicx}
\usepackage{amsmath}
\begin{document}
For equation
\[ax^2+bx+c=0\]
We have
\[x_{1,2}=\frac{-b\pm \sqrt{b^2-4ac}}{2a}\]
Many Dots
\[ x_1,x_2,\dots ,x_n\quad 1,2,\cdots ,n\quad
\vdots\quad \ddots \]
Dets
\[ \begin{pmatrix} a&b&g\\c&d&h\\e&f&i \end{pmatrix} \quad
\begin{bmatrix} a&b\\c&d \end{bmatrix} \quad
\begin{Bmatrix} a&b\\c&d \end{Bmatrix} \quad
\begin{vmatrix} a&b\\c&d \end{vmatrix} \quad
\begin{Vmatrix} a&b\\c&d \end{Vmatrix} \]
Case Function
\[ y= \begin{cases}
    -x,\quad x\leq 0 \\
    x,\quad x>0
    \end{cases} \]
Newton-Leibniz formula
\[\int_{a}^{b} f(x) \,dx=F(x)|^b_a=F(b)-F(a) \]
Chemistry
\[\Delta _rG_m^\ominus (T)=-RT\ln K^\ominus=-zFE^\ominus \]
For more common conditions
\[\Delta _rG_m(T)=\Delta _rG_m^\ominus+RT\ln Q\]
and we can easily calculate other temperature's by
\[\ln \frac{K_2^\ominus}{K_1^\ominus}=-\frac{\Delta_rH_m^\ominus}{R}\Big(\frac{1}{T_2}-\frac{1}{T_1}\Big)\]
because
\[\ln K^\ominus=\frac{-\Delta_rH_m^\ominus}{RT}+\frac{\Delta_rS_m^\ominus}{R}\]
Chemical Battery
\[E=\varphi (+)-\varphi (-)\]
\[\varphi=\varphi^\ominus+\frac{RT}{zF}\ln\frac{c(Oxidation\ state)/c^\ominus}{c(Reduction\ state)/c^\ominus}\]
\[(z=Trans\ e^-)\]
\[(For\ example:O_2+2H_2O+4e^-=4OH^-;z=4)\]
\newpage
%Question:
有电池$Ag|Ag^+(a_1)|Br^-(a_2)|AgBr(s)|Ag$,已知:AgBr(s)溶度积25℃时为$5\times10^{-13}$,$\varphi^\ominus(Ag^+/Ag)=0.799V, \varphi^\ominus(Br_2/Br_-)=1.065V$。\\
(1)写出此电池阳极和阴极表面的电极反应以及总电池反应。\\
(2)计算$Br^-|AgBr(s)|Ag$半电池反应的标准电极电势。\\
(3)计算$AgBr(s)$的标准生成Gibbs函数变$\Delta_rG_m^\ominus(AgBr(s),298.13K)$\\
Solve:\\(1)
阳极:$Ag-e^-=Ag^+$\\
阴极:$AgBr+e^-=Ag+Br^-$\\
总的:$AgBr(s)=Ag^++Br^-$\\
(2):即为阴极反应$AgBr+e^-=Ag+Br^-$\\
可知$ \varphi^\ominus(+)=E^\ominus+\varphi^\ominus(-)$\\
$ -zFE^\ominus=-RT\ln K^\ominus $\\
解得:$ \varphi^\ominus(+)=0.0713$ V\\
(3):构造原电池:$ Ag-e^-=Ag^+$(-)、$ \frac{1}{2}Br_2+e^-=Br^-(+)$、\\
总反应:$Ag+\frac{1}{2}Br_2=AgBr$\\
$\Delta_rG_m^\ominus(AgBr(s),298.13K)=-zFE^\ominus$\\
其中:$ E^\ominus=\varphi^\ominus(+)-\varphi^\ominus(-)=1.065-0.799=0.266V$\\
解得:$ \Delta_rG_m^\ominus(AgBr(s),298.13K)=-25.67$ KJ·mol$^{-1}$\\
\newpage
将40.0mL 0.10mol·L$^{-1}$ AgNO$_3$溶液和20.0mL 6.0mol·L$^{-1}$氨水混合并稀释至100mL。试计算:\\
(1)平衡时溶液中Ag$^+$、[Ag(NH$_3$)$_2$]$^+$和NH$_3$的浓度;\\
(2)加入0.010mol KCl固体,是否有AgCl沉淀产生?\\
(3)若要阻止AgCl沉淀产生,则应取12.0mol·L$^{-1}$氨水多少毫升?\\
Solve:\\
(1)
\newpage
Picture:\\
\centerline{\includegraphics[width = .4\textwidth]{Demo.jpg}}
\centerline{Picture 1: Zhongli\&Raiden Shogun are talking each other.}
\end{document}